\subsection{Relative rating functions}
First, the rating functions that belong to the relative comparison are described.
\subsubsection{Zipf's law distance}
\cite{Zipfslaw_paper} and \cite{Zipfslaw_paper2} described that songs that followed Zipf's law, or were closer to it than other songs, were more likely to be preferred by users. 
\\
A Zipfian distribution is a distribution where the 2nd highest occurring element, occurs 1/2 times the number of occurrences of the highest occurring element, the 3rd highest occurring element, occurs 1/3 times the number of occurrences of the highest occurring element and so on. Suppose L is the number of occurrences of the most popular element. For every element E, with rank R (position of the element on the list of unique elements ordered by its frequency of appearance), the amount of occurrences is defined by the following function:
\[ Occurences(E) =  \frac{1}{R} * L  \]
\\
The tendency of the song's elements to follow the Zipf's law is the concept for this rating function. The song's elements can be both intervals or notes. This rating function can be used in two ways. The candidate song can be rated on its tendency to follow the Zipf's law. The candidate song can also be rated based on how similar this tendency is between the master song and the candidate song.