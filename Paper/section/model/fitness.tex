
\subsection{The fitness function}
A composition has multiple aspects that can be rated at each individual level, therefore, the fitness of an individual is determined by different \textbf{rating functions}. Each of these rating functions gives a certain \textbf{score} for a particular \textbf{concept} of the song. We clarify with an example: the tendency to follow a musical scale within the composition can be such a concept. Every rating function calculates a score and to this score there is a predetermined weight attached that implies the importance of the rated concept. The total fitness of a song $x$ is equal to the sum of the products of all rating functions S for each concept and their corresponding weights W.
\[ TotalFitness(x) = \sum_{i=1}^{C} S_{i} * W_{i}\] 
where C is the number of concepts.


\[ S(x) =  difference( f(x) ,optimalscore) * S_{weight} \]

\subsubsection{The master song}
The fitness function calculates a score song based on the master. The master song is set during the initialisation process and it controls the population by defining the rules. We can rate candidate songs based on the master song in two ways:
\begin{itemize}
    \item absolute comparison: this is where we compare the elements of the master song directly with the candidate song.
    \item relative comparison: this is where we compare the relative structure of the master song directly with the candidate song.
\end{itemize}

