
\subsection{The fitness function}
A composition has multiple aspects that can be rated at an individual level as it should. Therefore, the fitness function can be divided into multiple sub-functions that belong to a concept. This concept can be seen as what it is that the function-gives a score, e.g. the tendency to follow an musical scale within the composition can be a concept. Every sub-function calculates a score and to this score there is a predetermined weight attached that implies the importance of the rated concept. The total fitness of a song x is equal to the sum of products of all sub-functions S an their corresponding weights.
\[ TotalFitness(x) = \sum_{S=1}^{n} S_{concept} * W_{concept} \   \text{ }\] 


\[ S(x) =  difference( f(x) ,optimalscore) * S_{weight} \]

\subsubsection{The master song}
The fitness function calculates a score song based on the master. The master song is set during the initialisation process and it controls the population by defining the rules. We can rate candidate songs based on the master song in two ways:
\begin{itemize}
    \item absolute comparison: this is where we compare the elements of the master song directly with the candidate song.
    \item relative comparison: this is where we compare the relative structure of the master song directly with the candidate song.
\end{itemize}

